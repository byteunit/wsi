{
    \def\pvala{0.5}
    \def\pvalb{0.25}
    \def\pvalc{0.75}
    \def\pvald{0.9}

    \pgfmathtruncatemacro{\maxval}{\nval}

    \begin{axis}[
        domain=-4:4,
        axis x line=bottom, % no box around the plot, only x and y axis
        axis y line=left, % the * would suppress the arrow tips
        legend pos=north west,
        xlabel={k},
        ylabel={p},%TODO: was ist diese Achse?
        samples=200,
        height=5cm,
        width=7cm,
        clip=false]

        \pgfkeys{/pgf/fpu=true}
        \pgfkeys{/pgf/number format/.cd,fixed,precision=5}

        \def\x{1}
        \def\nextx{2}
        %\foreach \x/\y [remember=\x as \lastx (initially A)] in {B/2,C/3,D/4,E/5}
        \pgfmathparse{binomial_distribution(\nval,\x,\pvalb)}
        \xdef\tvalb{\pgfmathresult}
        \addplot[blue,forget plot] coordinates{(\x,\tvalb)(\nextx,\tvalb)};
        \addplot[continuityblue] coordinates{(\x,\tvalb)};
        \addlegendentry[align=left]{\scriptsize{$\mathcal B(k|\nval,\pvalb)$}};

        \pgfmathparse{binomial_distribution(\nval,\x,\pvalc)}
        \xdef\tvalc{\pgfmathresult}
        \addplot[green,forget plot] coordinates{(\x,\tvalc)(\nextx,\tvalc)};
        \addplot[continuitygreen] coordinates{(\x,\tvalc)};
        \addlegendentry[align=left]{\scriptsize{$\mathcal B(k|\nval,\pvalc)$}};

        \pgfmathparse{binomial_distribution(\nval,\x,\pvald)}
        \xdef\tvald{\pgfmathresult}
        \addplot[orange,forget plot] coordinates{(\x,\tvald)(\nextx,\tvald)};
        \addplot[continuityorange] coordinates{(\x,\tvald)};
        \addlegendentry[align=left]{\scriptsize{$\mathcal B(k|\nval,\pvald)$}};

        \pgfmathparse{binomial_distribution(\nval,\x,\pvala)}
        \xdef\tvala{\pgfmathresult}
        \addplot[red,forget plot] coordinates{(\x,\tvala)(\nextx,\tvala)};
        \addplot[continuityred] coordinates{(\x,\tvala)};
        \addlegendentry[align=left]{\scriptsize{$\mathcal B(k|\nval,\pvala)$}};

        \foreach \x [remember=\x as \lastx (initially 0)] in {2,...,\maxval} {
            \pgfkeys{/pgf/fpu=false}
            \pgfmathtruncatemacro{\nextx}{\x+1}
            \pgfkeys{/pgf/fpu=true}
            \pgfkeys{/pgf/number format/.cd,fixed,precision=5}

            \pgfmathparse{\tvalb+binomial_distribution(\nval,\x,\pvalb)}
            \xdef\tvalb{\pgfmathresult}
            \addplot[blue,forget plot] coordinates{(\x,\tvalb)(\nextx,\tvalb)};
            \addplot[continuityblue,forget plot] coordinates{(\x,\tvalb)};

            \pgfmathparse{\tvalc+binomial_distribution(\nval,\x,\pvalc)}
            \xdef\tvalc{\pgfmathresult}
            \addplot[green,forget plot] coordinates{(\x,\tvalc)(\nextx,\tvalc)};
            \addplot[continuitygreen,forget plot] coordinates{(\x,\tvalc)};

            \pgfmathparse{\tvald+binomial_distribution(\nval,\x,\pvald)}
            \xdef\tvald{\pgfmathresult}
            \addplot[orange,forget plot] coordinates{(\x,\tvald)(\nextx,\tvald)};
            \addplot[continuityorange,forget plot] coordinates{(\x,\tvald)};

            \pgfmathparse{\tvala+binomial_distribution(\nval,\x,\pvala)}
            \xdef\tvala{\pgfmathresult}
            \addplot[red,forget plot] coordinates{(\x,\tvala)(\nextx,\tvala)};
            \addplot[continuityred,forget plot] coordinates{(\x,\tvala)};
        }
        \pgfkeys{/pgf/fpu=false}


    \end{axis}
}
