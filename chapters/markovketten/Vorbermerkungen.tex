% This file was converted to LaTeX by Writer2LaTeX ver. 1.0.2
% see http://writer2latex.sourceforge.net for more info
%\documentclass[a4paper]{article}
%\usepackage[ascii]{inputenc}
%\usepackage[LGR,T1]{fontenc}
%\usepackage[ngerman,greek,english,english]{babel}
%\usepackage{amsmath}
%\usepackage{amssymb,amsfonts,textcomp}
%\usepackage{color}
%\usepackage{array}
%\usepackage{hhline}
%\usepackage{hyperref}
%\hypersetup{pdftex, colorlinks=true, linkcolor=blue, citecolor=blue, filecolor=blue, urlcolor=blue, pdftitle=, pdfauthor=, pdfsubject=, pdfkeywords=}
% Page layout (geometry)
%\setlength\voffset{-1in}
%\setlength\hoffset{-1in}
%\setlength\topmargin{2cm}
%\setlength\oddsidemargin{2cm}
%\setlength\textheight{25.7cm}
%\setlength\textwidth{17.001cm}
%\setlength\footskip{0.0cm}
%\setlength\headheight{0cm}
%\setlength\headsep{0cm}
% Footnote rule
%\setlength{\skip\footins}{0.119cm}
%\renewcommand\footnoterule{\vspace*{-0.018cm}\setlength\leftskip{0pt}\setlength\rightskip{0pt plus 1fil}\noindent\textcolor{black}{\rule{0.25\columnwidth}{0.018cm}}\vspace*{0.101cm}}
% Pages styles
%\makeatletter
%\newcommand\ps@Standard{
%  \renewcommand\@oddhead{}
%  \renewcommand\@evenhead{}
%  \renewcommand\@oddfoot{}
%  \renewcommand\@evenfoot{}
%  \renewcommand\thepage{\arabic{page}}
%}
%\makeatother
%\pagestyle{Standard}
%\title{}
%\author{}
%\date{2014-02-24T15:50:54.234000000}
%\begin{document}
{\selectlanguage{english}\bfseries
Vorbemerkungen}


\bigskip

Um es gleich vorweg zu nehmen: wir betrachten hier Markovketten. Das
sind Markovprozesse (ein Spezialfall von stochastischen Prozessen) mit
diskretem Zustandsraum. D.h.: Die Zust\"ande in unserer Markovkette
k\"onnen wir als Folgen von Zufallsvariablen betrachten oder auch
anschaulicher als Knoten eines Graphen die wir \"uber einen Namen in
Form einer ganzen Zahl identifizieren k\"onnen. Beispielsweise k\"onnte
unsere Markovkette die Zust\"ande \ \{-3\}, \{-2\}, \{-1\}, \{0\},
\{1\}, \{2\} enthalten. Au{\ss}erdem besch\"aftigen wir uns hier
vorerst mit Markovketten nur in diskreter Zeit, was zu einer weiteren
Vereinfachung f\"uhrt. \ Wir pr\"ufen also nur zu diskreten Zeitpunkten
 $(0\le t\le \infty )\text{  bzw. }(0\le n\le \infty )$ in welchem
Zustand sich die Markovkette befindet.

\begin{bsp}
Ein Beispiel k\"onnte eine Spielfigur sein, welche jede Runde am
Spielbrett (beispielsweise dem von DKT) entlang zieht. Dies geschieht
abh\"angig vom W\"urfelergebnis. Wir nehmen an wir w\"urfeln mit nur
einem \ (fairen) W\"urfel. Wenn wir jede Runde pr\"ufen auf welchem
Feld die Spielfigur zu stehen kommt, entsprechen die Runden unseren
diskreten Zeitpunkten und die Spielfelder (welche wir \"uber eine
Nummer identifizieren k\"onnten) entsprechen unseren diskreten
Zust\"anden.
\end{bsp}

{\selectlanguage{english}
Wir machen Momentaufnahmen, in welchem Zustand sich die Markovkette
gerade befindet und bezeichnen diese mit  $X_{n}$. Dabei handelt es
sich um Zufallsvariablen, denn Zust\"ande werden von anderen
Zust\"anden aus, nur mit gewissen Wahrscheinlichkeiten erreicht. 
$X_{3}=12$ K\"onnte bedeuten, dass sich die Markovkette zum Zeitpunkt 3
im Zustand 12 befindet. So k\"onnte etwa die Spielfigur auf Feld 12
stehen.}

{\selectlanguage{english}
Wie wir erkennen k\"onnen, sind f\"ur Spielfigur aus nicht alle
Spielfelder erreichbar sondern nur die (von der aktuellen Position aus
gesehen) n\"achsten 6 Felder. Die Wahrscheinlichkeit f\"ur den
\"Ubergang zu diesem Feld bezeichnen wir als
 \textbf{\"Ubergangswahrscheinlichkeiten}\index{"Ubergangswahrscheinlichkeit@Übergangswahrscheinlichkeit}
$\mathit{pij}=P(\mathit{Xn}+1=j|\mathit{Xn}=i)$}

{\selectlanguage{english}
Also die Wahrscheinlichkeit, im n\"achsten Zug auf Feld/im Zustand j zu
landen wenn wir uns jetzt auf Feld/im Zustand i befinden. In unserem
Beispiel k\"onnten dies lauten:}

 $p_{12,13}=\frac{1}{6}$  $p_{12,14}=\frac{1}{6}$ 
$p_{12,15}=\frac{1}{6}$  $p_{12,16}=\frac{1}{6}$ 
$p_{12,17}=\frac{1}{6}$  $p_{12,18}=\frac{1}{6}$ 


\bigskip

{\selectlanguage{english}
Alle anderen Zust\"ande w\"aren nicht erreichbar, d.h.: die
entsprechenden \"Ubergangswahrscheinlichkeiten w\"aren 0. Wir k\"onnen
uns vorstellen, dass auch die \textbf{Markoveigenschaft}\index{Markoveigenschaft} zu trifft. Diese besagt,
dass die Wahrscheinlichkeit, einen bestimmten Zustand zu erreichen, nur
vom aktuellen Zustand abh\"angt und nicht von den Zust\"anden die zuvor
angenommen wurden. Auch f\"ur den n\"achsten Zug der Spielfigur wird es
unerheblich sein, welche Zust\"ande/Felder sie zuvor passiert hat --
erreichbar werden immer nur die von der aktuellen Position aus
gesehenen 6 Felder sein.}


\bigskip

{\selectlanguage{english}
Wir k\"onnen auch t-stufige Übergangswahrscheinlichkeiten definieren: 
$\mathit{pij}(t)=P(X_{n+t}=j|X_{n=i})$ }

{\selectlanguage{english}
Diese sind Wahrscheinlichkeiten, einen Zustand in t-Schritten bzw.
Z\"ugen anzunehmen. }

Wir fassen unsere \"Ubergangswahrscheinlichkeiten zu Matrizen P bzw.
P(t) zusammen. P(t) ist die \textbf{t-stufige \"Ubergangsmatrix}\index{"Ubergangsmatrix@Übergangsmatrix!t-stufige} und kann aus
den Potenzen  $P(t)=P^{t}$ berechnet werden.


\bigskip

{\selectlanguage{english}
    Ein weiterer wichtiger Begriff ist die \textbf{Wahrscheinlichkeitsverteilung}\index{Wahrscheinlichkeitsverteilung}.
    Als einfaches Beispiel nehmen wir hier die \textbf{Anfangsverteilung}\index{Anfangsverteilung} her. Diese
gibt f\"ur jeden Zustand i die Wahrscheinlichkeit an, dass sich die
Kette zu Beginn \ {}- also zum Zeitpunkt n=0 -- in Zustand i befindet.}

{\selectlanguage{english}
Diese (wie auch die anderen Verteilungen) wird in Form eines
Wahrscheinlichkeitsvektors angegeben, welcher als Zeilenvektor
aufgefasst wird. Nennen wir unsere Anfangsverteilung \gls{symb:alphaunder} hier z.B.: 
$\underline{{\alpha }}=(\begin{matrix}\alpha _{0}&\alpha
_{1}&...&\alpha _{k}\end{matrix})$}

{\selectlanguage{ngerman}
Dann ist \ diese in unserem Spiel-Beispiel:}

\begin{equation*}
\underline{{\alpha }}=(\begin{matrix}1&0&...&0\end{matrix})
\end{equation*}
{\selectlanguage{ngerman}
Wenn wir davon ausgehen, dass die Figur definitiv am Start-Feld startet
und dieses als Zustand 0 bezeichnet wird.}

{\selectlanguage{ngerman}
    Im folgenden wird auch von der \textbf{station\"aren Verteilung}\index{stationäre Verteilung} \gls{symb:piunder}
 die Rede sein, welche
inkonsistenter Weise auch oft ohne den Vektor-Unterstrich angegeben
wird. Wichtig ist zu wissen, dass diese ebenfalls durch einen
Wahrscheinlichkeitsvektor in Form eines Zeilenvektors beschrieben wird
und  $\pi _{i}$ die einzelnen Eintr\"age bezeichnet. }

{\selectlanguage{english}
\foreignlanguage{ngerman}{Eine
Beme}\foreignlanguage{ngerman}{rk}\foreignlanguage{ngerman}{ung ist
noch zu machen was }\foreignlanguage{ngerman}{Formeln mit Summen
betrifft. In den meisten F\"allen sind darin auch
\"Ubergangswahrscheinlichkeiten enthalten und wie wir sehen werden sind
dieses sehr oft 0. D.h.: Es fallen ohnehin die meisten Glieder der
Reihe weg und es bleibt meistens eine mehr oder weniger einfache
Differenzengleichung f\"ur die gesuchte Unbekannte \"ubrig. Mehr dazu
in den Beisp}\foreignlanguage{ngerman}{ielen, welche am Ende des
Kapitels folgen}\foreignlanguage{ngerman}{.}}

{\selectlanguage{ngerman}
Mit diesem Wissen, sollte es m\"oglich sein, das nachfolgende Skriptum
sowie die Beispiele einigerma{\ss}en zu verstehen, ohne dass einen die
Symbole all zu sehr irritieren.}




%\end{document}
