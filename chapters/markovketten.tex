\chapter{Markovketten}

% This file was converted to LaTeX by Writer2LaTeX ver. 1.0.2
% see http://writer2latex.sourceforge.net for more info
%\documentclass[a4paper]{article}
%\usepackage[ascii]{inputenc}
%\usepackage[LGR,T1]{fontenc}
%\usepackage[ngerman,greek,english,english]{babel}
%\usepackage{amsmath}
%\usepackage{amssymb,amsfonts,textcomp}
%\usepackage{color}
%\usepackage{array}
%\usepackage{hhline}
%\usepackage{hyperref}
%\hypersetup{pdftex, colorlinks=true, linkcolor=blue, citecolor=blue, filecolor=blue, urlcolor=blue, pdftitle=, pdfauthor=, pdfsubject=, pdfkeywords=}
% Page layout (geometry)
%\setlength\voffset{-1in}
%\setlength\hoffset{-1in}
%\setlength\topmargin{2cm}
%\setlength\oddsidemargin{2cm}
%\setlength\textheight{25.7cm}
%\setlength\textwidth{17.001cm}
%\setlength\footskip{0.0cm}
%\setlength\headheight{0cm}
%\setlength\headsep{0cm}
% Footnote rule
%\setlength{\skip\footins}{0.119cm}
%\renewcommand\footnoterule{\vspace*{-0.018cm}\setlength\leftskip{0pt}\setlength\rightskip{0pt plus 1fil}\noindent\textcolor{black}{\rule{0.25\columnwidth}{0.018cm}}\vspace*{0.101cm}}
% Pages styles
%\makeatletter
%\newcommand\ps@Standard{
%  \renewcommand\@oddhead{}
%  \renewcommand\@evenhead{}
%  \renewcommand\@oddfoot{}
%  \renewcommand\@evenfoot{}
%  \renewcommand\thepage{\arabic{page}}
%}
%\makeatother
%\pagestyle{Standard}
%\title{}
%\author{}
%\date{2014-02-24T15:50:54.234000000}
%\begin{document}
{\selectlanguage{english}\bfseries
Vorbemerkungen}


\bigskip

Um es gleich vorweg zu nehmen: wir betrachten hier Markovketten. Das
sind Markovprozesse (ein Spezialfall von stochastischen Prozessen) mit
diskretem Zustandsraum. D.h.: Die Zust\"ande in unserer Markovkette
k\"onnen wir als Folgen von Zufallsvariablen betrachten oder auch
anschaulicher als Knoten eines Graphen die wir \"uber einen Namen in
Form einer ganzen Zahl identifizieren k\"onnen. Beispielsweise k\"onnte
unsere Markovkette die Zust\"ande \ \{-3\}, \{-2\}, \{-1\}, \{0\},
\{1\}, \{2\} enthalten. Au{\ss}erdem besch\"aftigen wir uns hier
vorerst mit Markovketten nur in diskreter Zeit, was zu einer weiteren
Vereinfachung f\"uhrt. \ Wir pr\"ufen also nur zu diskreten Zeitpunkten
 $(0\le t\le \infty )\text{  bzw. }(0\le n\le \infty )$ in welchem
Zustand sich die Markovkette befindet.

\begin{bsp}
Ein Beispiel k\"onnte eine Spielfigur sein, welche jede Runde am
Spielbrett (beispielsweise dem von DKT) entlang zieht. Dies geschieht
abh\"angig vom W\"urfelergebnis. Wir nehmen an wir w\"urfeln mit nur
einem \ (fairen) W\"urfel. Wenn wir jede Runde pr\"ufen auf welchem
Feld die Spielfigur zu stehen kommt, entsprechen die Runden unseren
diskreten Zeitpunkten und die Spielfelder (welche wir \"uber eine
Nummer identifizieren k\"onnten) entsprechen unseren diskreten
Zust\"anden.
\end{bsp}

{\selectlanguage{english}
Wir machen Momentaufnahmen, in welchem Zustand sich die Markovkette
gerade befindet und bezeichnen diese mit  $X_{n}$. Dabei handelt es
sich um Zufallsvariablen, denn Zust\"ande werden von anderen
Zust\"anden aus, nur mit gewissen Wahrscheinlichkeiten erreicht. 
$X_{3}=12$ K\"onnte bedeuten, dass sich die Markovkette zum Zeitpunkt 3
im Zustand 12 befindet. So k\"onnte etwa die Spielfigur auf Feld 12
stehen.}

{\selectlanguage{english}
Wie wir erkennen k\"onnen, sind f\"ur Spielfigur aus nicht alle
Spielfelder erreichbar sondern nur die (von der aktuellen Position aus
gesehen) n\"achsten 6 Felder. Die Wahrscheinlichkeit f\"ur den
\"Ubergang zu diesem Feld bezeichnen wir als
 \textbf{\"Ubergangswahrscheinlichkeiten}\index{"Ubergangswahrscheinlichkeit@Übergangswahrscheinlichkeit}
$\mathit{pij}=P(\mathit{Xn}+1=j|\mathit{Xn}=i)$}

{\selectlanguage{english}
Also die Wahrscheinlichkeit, im n\"achsten Zug auf Feld/im Zustand j zu
landen wenn wir uns jetzt auf Feld/im Zustand i befinden. In unserem
Beispiel k\"onnten dies lauten:}

 $p_{12,13}=\frac{1}{6}$  $p_{12,14}=\frac{1}{6}$ 
$p_{12,15}=\frac{1}{6}$  $p_{12,16}=\frac{1}{6}$ 
$p_{12,17}=\frac{1}{6}$  $p_{12,18}=\frac{1}{6}$ 


\bigskip

{\selectlanguage{english}
Alle anderen Zust\"ande w\"aren nicht erreichbar, d.h.: die
entsprechenden \"Ubergangswahrscheinlichkeiten w\"aren 0. Wir k\"onnen
uns vorstellen, dass auch die \textbf{Markoveigenschaft}\index{Markoveigenschaft} zu trifft. Diese besagt,
dass die Wahrscheinlichkeit, einen bestimmten Zustand zu erreichen, nur
vom aktuellen Zustand abh\"angt und nicht von den Zust\"anden die zuvor
angenommen wurden. Auch f\"ur den n\"achsten Zug der Spielfigur wird es
unerheblich sein, welche Zust\"ande/Felder sie zuvor passiert hat --
erreichbar werden immer nur die von der aktuellen Position aus
gesehenen 6 Felder sein.}


\bigskip

{\selectlanguage{english}
Wir k\"onnen auch t-stufige Übergangswahrscheinlichkeiten definieren: 
$\mathit{pij}(t)=P(X_{n+t}=j|X_{n=i})$ }

{\selectlanguage{english}
Diese sind Wahrscheinlichkeiten, einen Zustand in t-Schritten bzw.
Z\"ugen anzunehmen. }

Wir fassen unsere \"Ubergangswahrscheinlichkeiten zu Matrizen P bzw.
P(t) zusammen. P(t) ist die \textbf{t-stufige \"Ubergangsmatrix}\index{"Ubergangsmatrix@Übergangsmatrix!t-stufige} und kann aus
den Potenzen  $P(t)=P^{t}$ berechnet werden.


\bigskip

{\selectlanguage{english}
    Ein weiterer wichtiger Begriff ist die \textbf{Wahrscheinlichkeitsverteilung}\index{Wahrscheinlichkeitsverteilung}.
    Als einfaches Beispiel nehmen wir hier die \textbf{Anfangsverteilung}\index{Anfangsverteilung} her. Diese
gibt f\"ur jeden Zustand i die Wahrscheinlichkeit an, dass sich die
Kette zu Beginn \ {}- also zum Zeitpunkt n=0 -- in Zustand i befindet.}

{\selectlanguage{english}
Diese (wie auch die anderen Verteilungen) wird in Form eines
Wahrscheinlichkeitsvektors angegeben, welcher als Zeilenvektor
aufgefasst wird. Nennen wir unsere Anfangsverteilung \gls{symb:alphaunder} hier z.B.: 
$\underline{{\alpha }}=(\begin{matrix}\alpha _{0}&\alpha
_{1}&...&\alpha _{k}\end{matrix})$}

{\selectlanguage{ngerman}
Dann ist \ diese in unserem Spiel-Beispiel:}

\begin{equation*}
\underline{{\alpha }}=(\begin{matrix}1&0&...&0\end{matrix})
\end{equation*}
{\selectlanguage{ngerman}
Wenn wir davon ausgehen, dass die Figur definitiv am Start-Feld startet
und dieses als Zustand 0 bezeichnet wird.}

{\selectlanguage{ngerman}
    Im folgenden wird auch von der \textbf{station\"aren Verteilung}\index{stationäre Verteilung} \gls{symb:piunder}
 die Rede sein, welche
inkonsistenter Weise auch oft ohne den Vektor-Unterstrich angegeben
wird. Wichtig ist zu wissen, dass diese ebenfalls durch einen
Wahrscheinlichkeitsvektor in Form eines Zeilenvektors beschrieben wird
und  $\pi _{i}$ die einzelnen Eintr\"age bezeichnet. }

{\selectlanguage{english}
\foreignlanguage{ngerman}{Eine
Beme}\foreignlanguage{ngerman}{rk}\foreignlanguage{ngerman}{ung ist
noch zu machen was }\foreignlanguage{ngerman}{Formeln mit Summen
betrifft. In den meisten F\"allen sind darin auch
\"Ubergangswahrscheinlichkeiten enthalten und wie wir sehen werden sind
dieses sehr oft 0. D.h.: Es fallen ohnehin die meisten Glieder der
Reihe weg und es bleibt meistens eine mehr oder weniger einfache
Differenzengleichung f\"ur die gesuchte Unbekannte \"ubrig. Mehr dazu
in den Beisp}\foreignlanguage{ngerman}{ielen, welche am Ende des
Kapitels folgen}\foreignlanguage{ngerman}{.}}

{\selectlanguage{ngerman}
Mit diesem Wissen, sollte es m\"oglich sein, das nachfolgende Skriptum
sowie die Beispiele einigerma{\ss}en zu verstehen, ohne dass einen die
Symbole all zu sehr irritieren.}




%\end{document}


\section{Stochastische Prozesse}
\begin{definition}Ein stochastischer Prozess ist 
eine Familie $(X_t,t\in T)$ von Zufallsvariablen. Die Indexmenge $T$
wird Parameterraum genannt und soll eine Teilmenge der reellen Zahlen sein.
Der Wertebereich $M_X$ von $X_t$ heißt Zustandsraum oder Phasenraum.
Wenn $T$ endlich oder abzählbar (etwa $\mathbb N$) 
ist, sprechen wir von einem Prozess in
diskreter Zeit, wenn $T$ ein ganzes (endliches oder unendliches) Intervall
ist, von einem Prozess in stetiger Zeit, 
\end{definition}
Stochastische Prozesse in diskreter Zeit sind einfach Folgen von
 Zufallsvariablen. Der Unterschied zu unseren früheren Überlegungen besteht 
darin, dass wir nicht mehr annehmen, dass die einzelnen Zufallsvariablen
unabhängig sind. Wir müssen also die Abhängigkeiten zwischen den einzelnen
Zufallsvariablen festlegen, das heißt, wir müssen gewisse Annahmen über
die gemeinsame Verteilung von $(X_{t_1},\dots,X_{t_n})$ mit 
$t_1<\dots<t_n$ treffen (Ein berühmter Satz von Kolmogorov besagt, dass
durch die Angabe dieser ``endlichdimensionalen Randverteilungen'' ein
stochastischer Prozess festgelegt wird). 
Einige Möglichkeiten zählen wir jetzt auf:
\begin{definition}
Der Prozess $(X_t, t\in T)$ heißt Prozess mit unabhängigen Zuwächsen,
wenn für $t_1<t_2<\dots>t_n$ die Zufallsvariablen
\[X(t_1), X(t_2)-X(t_1), X(t_3)-X(t_2),\dots, X(t_n)-X(t_{n-1})\]
unabhängig sind.
\end{definition}
Hier verwenden wir die Notation $X(t)$ für $X_t$, damit die Indizes nicht
zu sehr überladen werden. 
Prozesse mit unabhängigen Zuwächsen in diskreter Zeit sind einfach
Summen von unabhängigen Zufallsvariablen, wie wir sie im letzten Kapitel
untersucht haben.
\begin{definition}
Der Prozess $X(t)$ heißt Markovprozess, wenn für
wenn für $t_1<t_2<\dots>t_n$ 
\[\mathbb P(X(t_n)\le x_n|X(t_1)=x_1,\dots,X(t_{n-1})=x_{n-1})=
\mathbb P(X(t_n)\le x_n|X(t_{n-1})=x_{n-1}).\]
\end{definition}
Die Zukunft hängt also von der Vergangenheit nur über den letzten Wert
$X(t_{n-1})$ ab. Ein Beispiel für Markovprozesse sind Prozesse mit
unabhängigen Zuwächsen.
\begin{definition}
Der Prozess $X_t,t\in T$ heißt stationär, wenn
wenn für $t_1<t_2<\dots>t_n$ und $h>0$
die gemeinsame Verteilung von $(X(t_1),\dots,X(t_n))$
mit der von $(X(t_1+h),\dots,X(t_n+h))$ übereinstimmt.
\end{definition}
\section{Markovketten in diskreter Zeit}
\subsection{Übergangswahrscheinlichkeiten}
\label{sec:uebergangswahrscheinlichkeit}
\index{"Ubergangswahrscheinlichkeit@Übergangswahrscheinlichkeit}
%TODO
Markovprozesse mit diskretem Zustandsraum nennen wir Markovketten. Wir 
können noch zwischen Markovketten in diskreter und in stetiger Zeit
unterscheiden. Die Diskussion von Markovketten in stetiger Zeit
werden wir auf später verschieben. In beiden Fällen kann man die
diskrete Verteilung der einzelnen Zufallsvariablen durch ihre Wahrscheinlichkeitsfunktion beschreiben, deshalb erhält die Markoveigenschaft die besonders einfache Form
\[\mathbb P(X_n= x_n|X_1=x_1,\dots,X_n=x_n)=
\mathbb P(X_n=x_n|X_{n-1}=x_{n-1}).\]
Die Wahrscheinlichkeiten
\[\mathbb P(X_{n+1}=j|X_n=i)\]
nennen wir die Übergangswahrscheinlichkeiten der Markovkette. Wenn diese
nicht von $n$ abhängen, sprechen wir von einer homogenen Markovkette und
setzen
\[p_{ij}=\mathbb P(X_{n+1}=j|X_n=i)\]

\begin{definition}
Die Wahrscheinlichkeiten
\[p_{ij}(t)=\mathbb P(X_{n+t}=j|X_n=i)\]
nennen wir die \textbf{$t$-stufigen Übergangswahrscheinlichkeiten}. 
Aus dem Satz von der vollständigen Wahrscheinlichkeit erhalten wir
die \textbf{Chapman-Kolmogorov Gleichungen}\index{Chapman-Kolmogorov Gleichungen}\index{Kolmogorov-Chapman Gleichungen|see{Chapman-Kolmogorov Gleichungen}} 
\end{definition}
%TODO: nicht als eigenen satz oder definition oder so???
\[p_{ij}(s+t)=\sum_{k\in M_X}p_{ik}(s)p_{kj}(t).\]
in Matrixnotation mit
\[P(t)=(p_{ij}(t))_{M_X\times M_X}\]
lauten die Chapman-Kolmogorov
Gleichungen
\[P(t+s)=P(t)P(s),\]
und mit $P=P(1)$
\[P(t)=P^t.\]
Wir nennen $P$ die \textbf{Übergangsmatrix} und $P(t)$ die \textbf{t-stufige Übergangsmatrix}.
\index{"Ubergangsmatrix@Übergangsmatrix}\index{"Ubergangsmatrix@Übergangsmatrix!t-stufige}

Wir setzen zusätzlich $p_i(t)=\mathbb P(X_t=i)$ und $p(t)=(p_i(t),i\in M_X)$
(als Zeilenvektor). Wieder mit dem Satz von der vollständigen Wahrscheinlichkeit erhalten wir
\[p(t)=p(0)P^t.\]
Wie man am besten Matrixpotenzen berechnet, siehe Anhang \ref{sec:matrixpotenz}.
Durch $p(0)$ und $P$ werden alle endlichdimensionalen Verteilungen festgelegt.

In Hinkunft verwenden wir zur Abkürzung die Notationen
\[\mathbb P_i(A)=\mathbb P(A|X_0=i)\]
und
\[\mathbb E_i(Y)=\mathbb E(Y|X_0=i).\]
\subsection{Klasseneigenschaften}
\label{sec:klasseneigenschaften}
Wir definieren
\begin{definition}
    \label{def:nachfolger}\label{def:verbunden}\label{def:kommunizierend}
    \index{Nachfolger}
Der Zustand $j$ heißt \textbf{Nachfolger} von $i$ ($i\rightarrow j$), wenn es ein
$t\ge 0$ gibt, sodass $p_{ij}(t)>0$.

Wenn sowohl $i\rightarrow j$ als auch $j\rightarrow i$ gilt, dann heißen
$i$ und $j$ \textbf{verbunden} oder \textbf{kommunizierend}.\index{verbunden}\index{kommunizierend}
\end{definition}

Das Kommunizieren ist eine Äquivalenzrelation, wir können daher 
den Phasenraum in die Äquivalenzklassen zerlegen, die wir Rekurrenzklassen
oder kurz Klassen nennen. Gibt es nur eine Klasse (wenn also alle
Zustände miteinander kommunizieren), heißt die Markovkette \textbf{irreduzibel}\index{irreduzibel}.
\index{absorbierender Zustand}\index{Zustand, absorbierend|see{absorbierender Zustand}}
Ein Zustand mit $p_{ii}=1$ heißt \textbf{absorbierender Zustand}. Ein solcher Zustand
ist offensichtlich eine Klasse für sich (no pun intended).
\begin{definition}
    \label{def:klasseneigenschaft}
Eine Eigenschaft heißt \textbf{Klasseneigenschaft}\index{Klasseneigenschaft}, wenn sie entweder für alle
Zustände einer Klasse oder für  keinen gilt.
\end{definition}

Ein einfaches Beispiel einer Klasseneigenschaft ist die Periode:
\begin{definition}
    \label{def:periode}\index{Periode}
Die \textbf{Periode} eines Zustandes ist
\[d(i)=\mathrm{ggT}\{t\ge 0:p_{ii}(t)>0\}.\]

Sie definiert also den kleinsten möglichen Weg, um von Zustand $i$ wieder zum Zustand $i$ zurückzukehren.
\end{definition}
Etwas spannender ist die \textbf{Rekurrenz}\index{Rekurrenz}: dazu definieren wir zuerst
\[\tau_i=\inf\{t>0:X_t=i\},\]
die \textbf{Übergangszeit}\index{"Ubergangszeit@Übergangszeit} bzw.\ \textbf{Rückkehrzeit}\index{R"uckkehrzeit@Rückkehrzeit} (je nachdem, ob $X_0\neq i$ ist
oder nicht) nach $i$,
und
\[\nu_i=\#\{t>0:X_t=i\},\]
die Anzahl der Besuche in $i$.

\begin{satz}
Die folgenden Bedingungen sind äquivalent:
\begin{enumerate}
\item $\mathbb P_i(\tau_i<\infty)=1$,
\item $\mathbb P_i(\nu_i=\infty)=1$,
\item $\mathbb E_i(\nu_i)=\infty$,
\item $\sum_tp_{ii}(t)=\infty$.
\end{enumerate}
Wenn diese Bedingungen erfüllt sind, dann heißt $i$ rekurrent, sonst
transient. Rekurrenz und Transienz sind Klasseneigenschaften.
\end{satz}

Bei der Rekurrenz kann man weiter unterscheiden:
\begin{definition}
$i$ sei ein rekurrenter Zustand. Wenn 
\[\mathbb E_i(\tau_i)<\infty\]
gilt, dann heißt $i$ \textbf{positiv rekurrent}, sonst \textbf{nullrekurrent}.\index{Rekurrenz!positiv-|see{positiv Rekurrent}}\index{Rekurrenz!null-|see{Nullrekurrent}}\index{Nullrekurrent}\index{positiv Rekurrent}
\end{definition}

\begin{definition}
$(\pi_i, i\in M_X)$ heißt \textbf{stationäre Verteilung}\index{stationäre Verteilung}, wenn
\[\pi_i\ge 0,\]
\[\sum_i\pi_i=1\]
und
\[\sum_i \pi_ip_{ij}=p_j.\]

Die stationäre Verteilung ist die Verteilung, die man für die Zustände
einer Markovkette erhält, wenn man sie nach einer sehr langen Zeit betrachtet. Dann sollten die Anfangseffekte (= Anfangswahrscheinlichkeit)
nicht mehr sichtbar sein. Dazu müssen wir erst die Wahrscheinlichkeit für
einen Zustand $i$ zum Zeitpunkt $t$ berechen.
\end{definition}

\begin{satz}
Wenn $(X_n)$ irreduzibel und aperiodisch ist, dann existieren die
Grenzwerte
\[\lim_{n\to\infty}p_{ij}(t)=\pi_j={1\over \mathbb E_j(\tau_j)}.\]
Im periodischen Fall gilt
\[\pi_j=\lim_{n\to\infty}{1\over n}\sum_{t=1}^{n}p_{ij}(t).\]
Wenn $(\pi_i)$ nicht identisch verschwindet (also wenn die Kette positiv
rekurrent ist), dann ist es eine
\textbf{stationäre Verteilung}\index{stationäre Verteilung}. Umgekehrt folgt aus der Existenz
einer stationären Verteilung die positive Rekurrenz. Die positive Rekurrenz
bzw.\ Nullrekurrenz ist ebenfalls eine Klasseneigenschaft.
\end{satz}

Für einen absorbierenden Zustand $i_0$ definieren wir die 
\textbf{Absorptionswahrscheinlichkeit $a_i$}\index{Absorptionswahrscheinlichkeit} als
\[\mathbb P_i(\tau_{i_0}<\infty)=\mathbb P_i(X\mbox{wird in }i_0\mbox{ absorbiert}).\]



\begin{satz}
Die Absorptionswahrscheinlichkeiten sind die kleinste nichtnegative Lösung
des Gleichungssystems
\[a_{i_0}=1,\]
\[a_i=\sum_jp_{ij}a_j, i\neq i_0.\]
\end{satz}

Das gibt uns eine Möglichkeit, die Transienz oder Rekurrenz einer
irreduziblen Markovkette zu entscheiden. Wir wählen einen Zustand und
machen ihn zu einem absorbierenden Zustand und bestimmen für die
modifizierte Übergangsmatrix die Absorptionswahrscheinlichkeiten. Sind diese
gleich 1, ist die Kette rekurrent. 

Wir nehmen an, dass $i_0$ der einzige absorbierende Zustand ist und
alle anderen Zustände kommunizieren und die Absorptionswahrscheinlichkeiten
1 sind.
Dann erhalten wir für die mittlere Zeit bis zur Absorption (\textbf{mittlere Absorptionszeit}\index{mittlere Absorptionszeit}\index{Absorptionszeit!mittlere -|see{mittlere Absorptionszeit}})
\[m_i=\mathbb E_i(\tau_{i_0})\]
 eine ähnliche Gleichung:
\[m_{i_0}=0,
\]
\[m_i=1+\sum_jp_{ij}m_j, i\neq i_0.\]

Allgemeiner kann man eine Zufallsvariable 
$S=\sum_{t=1}^{\tau-1}x_{t,t+1}$ betrachten (d.h., wir ``sammeln'' auf
unserem Weg zur Absorption Beträge $x_{ij}$, die von dem jeweiligen
Übergang abhängen dürfen). 

Wir setzen 
\[m_i=\mathbb E_i(S)\]
und
\[v_i=\mathbb V_i(S).\]

Dann sind $m_i$ und $v_i$ die Lösungen der Gleichungen

\[m_{i_0}=0,v_i(i_0)=0,\]
\[m_i=\sum_jp_{ij}x_{ij}+\sum_jp_{ij}m_j, i\neq i_0.\]
\[v_i=\sum_jp_{ij}(x_{ij}i+m_j-m_i)^2+\sum_jp_{ij}v_j, i\neq i_0.\]

Wenn es mehrere absorbierende Zustände Zustände gibt, $A=\{i_1,\dots,i_k\}$,
dann kann man die Wahrscheinlichkeiten der Absorption  $a_i(i_j)$ in dem
absorbierenden Zustand $i_j$ oder allgemeiner in einem der Zustände in der
Menge $B\subseteq A$ ($a_i=a_i(B)$). Das gibt die Gleichungen
\[a_i=\sum_jp_{ij}a_j, i\not\in A\]
\[a_i=1, i\in B\]
\[a_i=0, i\in A\setminus B.\]

Manchmal ist es interessant (wieviel Geld werde ich haben, wenn ich nicht
bankrott gehe?), den Erwartungswert von $S$ unter der Bedingung, dass die
Absorption in einem bestimmten Zustand oder in einer bestimmten Teilmenge
der Menge der absorbierenden Zustände erfolgt. 
Da hilft es, dass $X(t)$ unter der Bedingung der Absorption in $B$
wieder eine Markovkette bildet, mit den Übergangswahrscheinlichkeiten
\[p_{ij}^B={p_{ij}a_j(B)\over a_i(B)}.\]
Mit diesen modifizierten Übergangswahrscheinlichkeiten (wobei die
absorbierenden Zustände $\not\in B$ weggelassen werden) kann man
die Gleichungen für $m_i$ $v_i$ lösen und erhält so die bedingte
Erwartung bzw.\ Varianz.




\subsection{Markov Chain Monte Carlo}



